\documentclass[12pt,oneside]{article}

\usepackage[russian]{babel}
\usepackage[utf8]{inputenc}
\usepackage{amsmath}
\usepackage{amssymb}
\usepackage{graphicx}
\usepackage{multicol}
\usepackage[14pt]{extsizes}
\usepackage[linesnumbered,boxed]{algorithm2e}
\usepackage{amsthm}
\usepackage{listings}
\usepackage{epstopdf}


\binoppenalty = 10000
\relpenalty = 10000
\textheight = 23cm
\textwidth = 17cm
\oddsidemargin = 0pt
\topmargin = -1.5cm
\parskip = 0pt
\tolerance = 2000
\flushbottom

\pagestyle{myheadings}

\begin{document}

\section{Постановка экспериментов}

\begin{enumerate}
\item Хотим подтвердить экспериментально, что скачки Адама с батч-нормализацией уменьшаются при увеличении эпсилон, а также сравнить сглаженную версию Адама с батч-нормализацией со стандартной версией Адама. Для этого мы запустим следующие эксперименты на датасете MNIST на трехслойной полносвязной сети:
\begin{itemize}
\item BN Adam со стандартным $\epsilon = 10^{-8}$ и сохраним параметры $v_t$ (3 раза)
\item BN Adam с увеличенным $\epsilon = 10^{-4}$ и сохраним параметры $v_t$ (3 раза)
\item Adam со стандартным $\epsilon = 10^{-8}$
\end{itemize}

\item Хотим провести эксперименты с дисперсией градиента и может быть подтвердить, что BN уменьшает дисперсию. Для этого мы запустим $K$ раз SGD и BN SGD на датасете MNIST для 3х- и 10ти-слойной полносвязных сетей из одного начального приближения. На каждой итерации будем записывать угол (или косинус угла) отклонения текущего стохастического шага от полного градиента в текущей точке. Затем усредним все отклонения и проверим, доминирует ли "дисперсия" SGD над "дисперсией" BN SGD.

\item Хотим убедиться, что на трехслойной сети на данных cluttered MNIST метод Адам не успевает обучиться за 50 эпох, для чего мы запустим уже существующий эксперимент на большее количество эпох.

\item Хотим проверить гипотезы об \emph{эпсилон} и \emph{улучшении методов при добавлении BN} для другой архитектуры сети, а именно сверточной (2 сверточных слоя с макс-пулингом, а затем 1 скрытый полносвязный слой). Для этого мы запустим следующие экперименты:
\begin{itemize}
\item Все методы с их BN версией на датасете MNIST
\item Все методы с их BN версией на датасете CIFAR-10
\item Метод BN Adam с $\epsilon_2 = 10^{-4}$ на обоих датасетах
\end{itemize}

\end{enumerate}


\end{document}
